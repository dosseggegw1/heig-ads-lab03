\documentclass[a4paper,11pt]{article}
\usepackage[T1]{fontenc}
\usepackage[utf8]{inputenc}
\usepackage{lmodern}
\usepackage[francais]{babel}
\usepackage{fullpage}
\usepackage{listings}
\usepackage{dirtree}
\lstdefinestyle{ascii-tree}{
  literate={├}{|}1 {─}{--}1 {└}{+}1 
}

\title{ADS Lab 03 - Pipelines}
\date{10 mars 2021}
\author{Gabriel Roch \and Gwendoline Dössegger}
\newcounter{commande}[subsection]
\newcommand{\question}[1]{\addtocounter{commande}{1}\paragraph{Question \arabic{commande}}#1\;}
\begin{document}

\maketitle
\section{Task1 - Exercices ON Redirection}
\question{Run the following commands and tell where stdout and stderr are redirected to.}
\begin{enumerate}
  \item \texttt{./out > file}       \\ Stdout : est redirigé dans file \\Stderr: est redirigé dans le terminal
  \item \texttt{./out 2> file}      \\ Stdout : est redirigé dans le terminal \\Stderr: est redirigé dans le fichier
  \item \texttt{./out 2>\&1 > file} \\ Stdout : est redirigé dans file\\Stderr: est redirigé dans le terminal
  \item \texttt{./out > file 2>\&1} \\ Stdout et Stderr sont redirigés dans file
\end{enumerate}

  
\question{What do the following commands do ?}
\begin{enumerate}
  \item \texttt{cat /usr/share/doc/bash/README | grep –i posix} \\L'objectif était d'afficher toutes les lignes contenant \texttt{posix} indépendanment de la casse présent dans le fichier README. Cependant, le fichier n'existant pas, grep n'affiche rien mais cat indique que le fichier n'existe pas.
  \item \texttt{ ./out 2>\&1 | grep –i eeeee} \\Cherche \texttt{eeeee} dans la sortie de .out/ (stderr et stdout). Rien n'est trouvé car la sortie est une altérnence de OE et donc aucune suite de \texttt{eeeee}. La recherche se fait en ignorant la casse.
  \item \texttt{./out 2>\&1 >/dev/null | grep –i eeeee} \\Cherche \texttt{eeeee} dans la sortie d'erreur de out (stderr). La ligne est trouvée et affichée dans le terminal \texttt{EEEEE}. La recherche se fait en ignorant la casse.
\end{enumerate}


\question{Write commands to perform the following tasks :}
\begin{enumerate}
  \item Produce a recursive listing, using \texttt{ls}, of files and directories in your home directory, including hidden files, in the file \texttt{/tmp/homefiles}. \\\texttt{ls --all -R \textasciitilde{} > /tmp/homefiles}\\
  
  
  \item Produce a (non-recursive) listing of all files in your home directory whose names end in .txt, .md or .pdf, in the file /tmp/documents. The command must not display an error message if there are no corresponding files. \\Nous avons trouvé les deux solutions suivantes :\\- \texttt{find \textasciitilde{} -maxdepth 1 -name '*.txt' -or -name '*.md' -or -name '*.pdf'} \\- \texttt{ls --all \textasciitilde/*.\{md,txt,pdf\} 2> /dev/null}
  
\end{enumerate}



\section{Task2 - LOG ANALYSIS}
Nous pouvons constaté qu'avec la commande \texttt{xxd}, que le séparateur des champs de la première ligne est le caractère 0x09 qui correspond à la tabulation.
\\
\begin{enumerate}
  \item How many log entries are in the file?\\ 2781 lignes \\Commande utilisée : \texttt{wc -l ads\_website.log}
  \item How many accesses were successful (server sends back a status of 200) and how many had an error of "Not Found" (status 404)?\\
  Nous affichons le nombre de fois que les codes de status du serveur apparaissent dans les log avec la commande suivante : \texttt{cut ads\_website.log -f10 | sort | uniq -c}\\
  - Statut 200, 1610 occurences\\
  - Statut 404, 21 occurences
  
  \item What are the URIs that generated a "Not Found" response? Be careful in specifying the correct search criteria: avoid selecting lines that happen to have the character sequence 404 in the URI.\\
  \texttt{cut ads\_website.log -f9-10 | grep "404\$" | sort -u | cut -f1 | tr '"' ' ' | cut -d ' ' -f3}\\
  
  
  \item How many different days are there in the log file on which requests were made?\\
  \texttt{cut -f3 ads\_website.log | cut -c2-12 | sort -u | wc -l}
  
  \item How many accesses were there on 4th March 2014? \\Nous avons trouvé les deux commandes suivantes : \\
  - \texttt{cut -f3 ads\_website.log | cut -c2-12 | grep "04/Mar/2014" | wc -l}\\
  - \texttt{cut -f3 ads\_website.log | cut -c2-12 | sort | uniq -c | grep "04/Mar/2014"}
  
  \item Which are the three days with the most accesses? Hint: Create first a pipeline that produces a list of dates preceded by the count of log entries on that date.\\
  \texttt{cut -f3 ads\_website.log | cut -c2-12 | sort | uniq -c | sort -gr | head -3}
  
  \item Which is the user agent string with the most accesses?
  \item If a web site is very popular and accessed by many people the user agent strings appearing in the server's log can be used to estimate the relative market share of the users' computers and operating systems. How many accesses were done from browsers that declare that they are running on Windows, Linux and Mac OS X?
  \item Read the documentation for the tee command. Repeat the analysis of the previous question for browsers running on Windows and insert tee into the pipeline such that the user agent strings (including repeats) are written to a file for further analysis (the filename should be useragents.txt). What are the operating systems Windows NT 6.1, 6.2 and 6.3?
\end{enumerate}


\section{Task3 - Conversion to CSV}


\end{document}
